\documentclass{article}

\usepackage{amsmath,amsthm}

\newcommand{\Ex}[1]{\subsection*{Exercise #1}}

\title{Reading Log of \\ \emph{Types and Programming Languages}}
\author{Seiya Tokui}

\begin{document}
\maketitle

\section{Introduction}

nothing.

\section{Mathematical Preliminaries}

\subsection{Ordered Sets}

\Ex{2.2.4}

$R'$ is reflexive by definition, so let us show that any reflexive relation $R''$
  that contains $R$ also contains $R'$.
Let $(s, t) \in R'$.
If $s = t$, then it holds that $(s, t) = (s, s) \in R''$ as $R''$ is reflexive.
Otherwise, $(s, t) \in R \subseteq R''$. \qed

\Ex{2.2.7}

Let $s, t, u$ satisfy $(s, t), (t, u) \in R^+ = \bigcup_i R_i$.
Since $(R_i)_i$ is a non-decreasing sequence of relations,
  there exists $k$ that satisfies $(s, t), (t, u) \in R_k$.
It then holds that $(s, u) \in R_{k + 1} \subseteq R^+$.
Therefore, the relation $R^+$ is transitive.

Let $R'$ be a transitive relation that contains $R$.
We show $R_i \subseteq R'$ by induction on $i$.
By definition, $R_0 = R \subseteq R'$.
Assume $R_i \subseteq R'$ for some $i$.
Let $(s, u) \in R_{i + 1}$.
If $(s, u) \in R_i$, then $(s, u) \in R'$.
Otherwise, there exists $t$ such that $(s, t), (t, u) \in R_i$.
Then, $(s, t), (t, u) \in R'$, which implies $(s, u) \in R'$ by transitivity of $R'$.
Therefore, $R_{i + 1} \subseteq R'$.
By taking the union, it holds that $R^+ = \bigcup_i R_i \subseteq R'$. \qed

\Ex{2.2.8}

W.l.o.g., we can assume $R$ is reflexive by replacing it with its reflexive closure.
Suppose that $R^\star$ is constructed from $R$ as $R^+$ in Exercise 2.2.7.
We show by induction on $i$ that $P$ is preserved by $R_i$.
First, $P$ is preserved by $R_0 = R$.
Assume that $P$ is preserved by $R_i$ for some $i$.
If $s \in P$ and $(s, u) \in R_{i + 1}$,
  it holds that $(s, u) \in R_i$
  or there exists $t$ such that $(s, t), (t, u) \in R_i$.
By inductive hypothesis,
  in the former case, $u \in P$,
  and in the latter case, $t \in P$ and thus $u \in P$.
Therefore, $P$ is preserved by $R_{i + 1}$, concluding the induction.
It then holds that $P$ is preserved by $R^\star = \bigcup_i R_i$. \qed

\end{document}
