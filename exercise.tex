\documentclass{article}

\usepackage{amsmath,amssymb,amsthm}
\usepackage{bussproofs}

\newcommand{\code}[1]{\texttt{#1}}
\newcommand{\Bigstep}{\Downarrow}
\newcommand{\Ex}[1]{\subsection*{Exercise #1}}
\newcommand{\False}{\code{false}}
\newcommand{\IfThenElse}[3]{\code{if}~#1~\code{then}~#2~\code{else}~#3}
\newcommand{\IsZero}[1]{\code{iszero}~#1}
\newcommand{\N}{\mathbb N}
\newcommand{\Pred}[1]{\code{pred}~#1}
\newcommand{\Rule}[1]{\textsc{#1}}
\newcommand{\Succ}[1]{\code{succ}~#1}
\newcommand{\True}{\code{true}}
\newcommand{\Wrong}{\code{wrong}}
\newcommand{\Zero}{\code0}

\title{Reading Log of \\ \emph{Types and Programming Languages}}
\author{Seiya Tokui}

\begin{document}
\maketitle

\section{Introduction}

nothing.

\section{Mathematical Preliminaries}

\Ex{2.2.4}

$R'$ is reflexive by definition, so let us show that any reflexive relation $R''$
  that contains $R$ also contains $R'$.
Let $(s, t) \in R'$.
If $s = t$, then it holds that $(s, t) = (s, s) \in R''$ as $R''$ is reflexive.
Otherwise, $(s, t) \in R \subseteq R''$. \qed

\Ex{2.2.7}

Let $s, t, u$ satisfy $(s, t), (t, u) \in R^+ = \bigcup_i R_i$.
Since $(R_i)_i$ is a non-decreasing sequence of relations,
  there exists $k$ that satisfies $(s, t), (t, u) \in R_k$.
It then holds that $(s, u) \in R_{k + 1} \subseteq R^+$.
Therefore, the relation $R^+$ is transitive.

Let $R'$ be a transitive relation that contains $R$.
We show $R_i \subseteq R'$ by induction on $i$.
By definition, $R_0 = R \subseteq R'$.
Assume $R_i \subseteq R'$ for some $i$.
Let $(s, u) \in R_{i + 1}$.
If $(s, u) \in R_i$, then $(s, u) \in R'$.
Otherwise, there exists $t$ such that $(s, t), (t, u) \in R_i$.
Then, $(s, t), (t, u) \in R'$, which implies $(s, u) \in R'$ by transitivity of $R'$.
Therefore, $R_{i + 1} \subseteq R'$.
By taking the union, it holds that $R^+ = \bigcup_i R_i \subseteq R'$. \qed

\Ex{2.2.8}

W.l.o.g., we can assume $R$ is reflexive by replacing it with its reflexive closure.
Suppose that $R^*$ is constructed from $R$ as $R^+$ in Exercise 2.2.7.
We show by induction on $i$ that $P$ is preserved by $R_i$.
First, $P$ is preserved by $R_0 = R$.
Assume that $P$ is preserved by $R_i$ for some $i$.
If $s \in P$ and $(s, u) \in R_{i + 1}$,
  it holds that $(s, u) \in R_i$
  or there exists $t$ such that $(s, t), (t, u) \in R_i$.
By induction hypothesis,
  in the former case, $u \in P$,
  and in the latter case, $t \in P$ and thus $u \in P$.
Therefore, $P$ is preserved by $R_{i + 1}$, concluding the induction.
It then holds that $P$ is preserved by $R^* = \bigcup_i R_i$. \qed

\section{Untyped Arithmetic Expressions}

\Ex{3.2.4}

We have $|S_0| = 0$ and $|S_{i + 1}| = 3 + 3|S_i| + |S_i|^3$,
  we thus get $|S_3| = 59439$.

\Ex{3.2.5}

We show it by induction on $i$.
We have $S_0 = \emptyset \subseteq S_1$.
Suppose $S_{i - 1} \subseteq S_i$ for given $i \in \{1, 2, \dots\}$.
Then, by definition of $S_{i + 1}$, we have
\begin{alignat*}{2}
  S_{i + 1} &=~
  && \{ \True, \False, \Zero \} \\
  & && \cup \{
    \Succ{\code t_1}, \Pred{\code t_1}, \IsZero{\code t_1}
    \mid \code t_1 \in S_i
  \} \\
  & && \cup \{
    \IfThenElse{\code t_1}{\code t_2}{\code t_3}
    \mid \code t_1, \code t_2, \code t_3 \in S_i
  \} \\
  &\supseteq~
  && \{ \True, \False, \Zero \} \\
  & && \cup \{
    \Succ{\code t_1}, \Pred{\code t_1}, \IsZero{\code t_1}
    \mid \code t_1 \in S_{i - 1}
  \} \\
  & && \cup \{
    \IfThenElse{\code t_1}{\code t_2}{\code t_3}
    \mid \code t_1, \code t_2, \code t_3 \in S_{i - 1}
  \} \\
  &=&& S_i.
\end{alignat*}
\qed

\Ex{3.3.4}

\paragraph{Induction on depth or size}
Let $Q(n) = \{ \code s \in S \mid depth(\code s) \le n \}$.
We have $S = \bigcup_{n\in\N} Q(n)$.
Induction on depth implies that,
  given $P(\code s)$ holds for all $\code s \in Q(n)$,
  $P(\code s)$ holds for all $\code s \in Q(n + 1)$.
By induction on $n$, $P(\code s)$ holds for all $\code s \in S$.
Induction on size can be derived by replacing $depth$ with $size$.

\paragraph{Structural induction}
Given $P(\code s)$ holds for all $\code s \in S_i$,
  we have $P(\code t)$ for
\begin{alignat*}{2}
  \code t &
  \in && ~\{
    \Succ{\code s}, ~\Pred{\code s}, \IsZero{\code s}
    \mid \code s \in S_i
  \} \\
  & &&\cup \{
    \IfThenElse{\code s_1}{\code s_2}{\code s_3}
    \mid \code s_1, \code s_2, \code s_3 \in S_i
  \}
\end{alignat*}
by induction hypothesis.
Together with constant terms, which all satisfy $P$ as they contain no subterms,
  we have $P(\code t)$ for all $\code t \in S_{i + 1}$.
We conclude that $P(\code s)$ holds for all $\code s \in S$ by induction on $i$.
\qed

\Ex{3.5.5}

If, for each derivation $D$, \\
\quad given $P(D')$ holds for all immediate subderivations $D'$ of $D$ \\
\quad we have $P(D)$, \\
then $P(D)$ holds for all $D$.

\Ex{3.5.10}

\begin{center}
  \AxiomC{$\code t \to^* \code t$}
  \DisplayProof
  \quad
  \AxiomC{$\code t \to \code t'$}
  \UnaryInfC{$\code t \to^* \code t'$}
  \DisplayProof
  \quad
  \AxiomC{$\code t \to^* \code t'$}
  \AxiomC{$\code t' \to^* \code t''$}
  \BinaryInfC{$\code t \to^* \code t''$}
  \DisplayProof
\end{center}

\Ex{3.5.13}

\begin{enumerate}
  \item Theorems 3.5.7, 3.5.8, and 3.5.12 hold.
  \item Theorems 3.5.7, 3.5.8, 3.5.11, and 3.5.12 hold.
        The proof of Theorem 3.5.11 should be changed to not rely on Theorem 3.5.4
          (which does not hold anymore).
\end{enumerate}

\Ex{3.5.14}

We use induction on derivation as is done in Theorem 3.5.4.
Suppose we have $\code t \to \code t'$ and $\code t \to \code t''$.
If $\code t$ is a conditional expression,
  then we repeat the discussion in the proof of Theorem 3.5.4
  to get $\code t' = \code t''$ using the induction hypothesis.
Otherwise, $\code t$ is either a successor, predecessor, or zero test.

If $\code t = \Succ{\code s}$ for some term $\code s$,
  the last rule used to derive both $\code t \to \code t'$
  and $\code t \to \code t''$ is \Rule{E-Succ}.
We then have $\code t' = \Succ{\code s'}$ for some term $\code s'$,
  and the statement $\Succ{\code s} \to \Succ{\code s'}$ is derived from
  $\code s \to \code s'$.
Similarly, we have $\code t'' = \Succ{\code s''}$ for some term $\code s''$,
  and the statement $\Succ{\code s} \to \Succ{\code s''}$ is derived from
  $\code s \to \code s''$.
We now have $\code s \to \code s'$ and $\code s \to \code s''$,
  and then we get $\code s' = \code s''$ by induction hypothesis.
It implies $\code t' = \Succ{\code s'} = \Succ{\code s''} = \code t''$.

If $\code t = \Pred{\code s}$ for some term $\code s$,
  the last rules used to derive
  $\code t \to \code t'$ and $\code t \to \code t''$ are
  either \Rule{E-PredZero}, \Rule{E-PredSucc}, or \Rule{E-Pred}.
These rules do not \emph{overlap},
  i.e., if $\code t \to \code t'$ is derived from one of the rules,
  then $\code t \to \code t''$ is also derived from the same rule.
To see this,
  we need to check if the left hand side $\Pred{\code t_1}$ of the statement
  derived by \Rule{E-Pred} does not conflict with the left hand side
  $\Pred{\Succ{\code{nv}_1}}$ of \Rule{E-PredSucc}.
Assume they conflict.
We then, for some $\code t'_1$,
  have $\Pred{\Succ{\code{nv}_1}} \to \Pred{\code t'_1}$
  derived by \Rule{E-Pred}.
The premise of this one-step derivation is $\Succ{\code{nv}_1} \to \code t'_1$.
The rule \Rule{E-Succ} is the only rule that may derive this statement,
  with which we have $\code{nv}_1 \to \code t'_2$ for some $\code t'_2$.
It contradicts with the fact that $\code{nv}_1$ is a normal form.
Therefore, we have shown that these three rules do not conflict with eath other.
Within these cases,
  \Rule{E-PredZero} and \Rule{E-PredSucc} both have a numeric value
  on the right hand side of the derived statement,
  and thus we have $\code t' = \code t''$.
For \Rule{E-Pred},
  we repeat the same discussion as \Rule{E-Succ} to get
  $\code t' = \code t''$ by induction hypothesis.

If $\code t = \IsZero{\code s}$ for some term $\code s$,
  a similar discussion is applied as the previous case of $\code t = \Pred{\code s}$,
  where the three rules are replaced with
  \Rule{E-IsZeroZero}, \Rule{E-IsZeroSucc}, and \Rule{E-IsZero},
  and we handle boolean values instead of numeric ones in some places. \qed

\Ex{3.5.16}

\begin{itemize}
  \item[(1)]
    \textsc{Proposition}:
    Let $\code t$ be any arithmetic expression.
    If $\code t$ is evaluated to a value,
      then it is evaluated to the same value in the augmented language.
    The term $\code t$ is evaluated to a stuck state
      iff it is evaluated to $\Wrong$ in the augmented language.
  \item[(2)] 
    \textsc{Proof}:
    Note first that the augmented evaluation rules are deterministic,
      since the added rules do not conflict with existing ones and with each other.
    Since the augmented language accomodates all the evaluation rules that
      the original language have,
      any expression evaluated to a value in the original language
      is evaluated by the same derivation to the same value in the augmented one.
    Assume $\code t$ is evaluated to a stuck state in the original language.
    Then, the term $\code t$ is in an either form of the left hand side of
      the added rules.
    Therefore, it is evaluated to $\Wrong$ in the augmented language.
    Conversely, assume $\code t$ is evaluated to $\Wrong$ in the augmented language.
    Since $\code t$ is an arithmetic expression,
      it is not $\Wrong$ itself,
      and thus is the result of an evaluation by one of the added rules.
    The left hand side of each added rule is a normal form in the original language,
      and thus is a stuck state. \qed
\end{itemize}

\Ex{3.5.17}

First, we show $\code t \Bigstep \code v$ given $\code t \to^* \code v$
  by induction on the number of small steps consisting $\code t \to^* \code v$.
If there are zero steps, i.e., $\code t = \code v$,
  then it holds that $\code t \Bigstep \code v$ by \Rule{B-Value}.
Consider the general case.
By induction hypothesis,
  there exists a term $\code t'$ that satisfies
  $\code t \to \code t'$,
  $\code t' \to^* \code v$,
  and $\code t' \Bigstep \code v$.
Then $\code t \Bigstep \code v$ is shown by examining all the cases for
  the direct derivation of $\code t \to \code t'$.
For example, if it is derived by \Rule{E-IfTrue},
  $\code t$ is written as $\code t = \IfThenElse{\True}{\code t'}{\code t''}$.
We have $\True \Bigstep \True$ by \Rule{B-Value}
  and $\code t' \Bigstep \code v$ by the hypothesis.
By applying \Rule{B-IfTrue} to them,
  we obtain $\IfThenElse{\True}{\code t'}{\code t''} \Bigstep \code v$.
Another example: if it is derived by \Rule{E-If},
  $\code t$ and $\code t'$ are written as
  $\code t = \IfThenElse{\code t_1}{\code t_2}{\code t_3}$ and
  $\code t' = \IfThenElse{\code t'_1}{\code t'_2}{\code t'_3}$, respectively.
We also have $\code t_1 \to \code t'_1$.
Since $\code t' \to^* \code v$,
  it holds that $\code t'_1 \to^* \code v_1$ with $\code v_1 \in \{\True, \False\}$.
We thus obtain $\code t_1 \to^* \code v_1$.
Suppose $\code v_1 = \True$.
Then, in the same way, we obtain $\code t_2 \to^* \code v$.
By induction hypothesis,
  we have $\code t_1 \Bigstep \True$ and $\code t_2 \Bigstep \code v$.
By applying \Rule{B-IfTrue},
  we get $\IfThenElse{\code t_1}{\code t_2}{\code t_3} \Bigstep \code v$.
Almost same discussions apply for the case of $\code v_1 = \False$.
The other cases also follow in a similar manner.

Next, we show $\code t \to^* \code v$ given $\code t \Bigstep \code v$
  by induction on derivation of $\code t \Bigstep \code v$.
When $\code t = \code v$, it obviously holds that $\code v \to^* \code v$.
Induction step is made by examining all the evaluation rules except for \Rule{B-Value}.
For example, if the last step is \Rule{B-IfTrue},
  we have $\code t = \IfThenElse{\code t_1}{\code t_2}{\code t_3}$
  with $\code t_1 \Bigstep \True$ and $\code t_2 \Bigstep \code v$.
By induction hypothesis,
  we have $\code t_1 \to^* \True$ and $\code t_2 \to^* \code v$.
By applying \Rule{E-If} multiple times with premises given by
  individual small steps consisting $\code t_1 \to^* \True$,
  we obtain
  $
    \IfThenElse{\code t_1}{\code t_2}{\code t_3} \to^*
    \IfThenElse{\True}{\code t_2}{\code t_3}
  $.
Next, we apply \Rule{E-IfTrue} to the right hand side to obtain
  $\IfThenElse{\code t_1}{\code t_2}{\code t_3} \to^* \code t_2$.
Combining it with $\code t_2 \to^* \code v$,
  we get $\IfThenElse{\code t_1}{\code t_2}{\code t_3} \to^* \code v$.
Similar discussions are applied to the other cases. \qed

\Ex{3.5.18}

The evaluation rules involving $\code{if}$ expression is modified as follows.
\begin{center}
  \AxiomC{$\IfThenElse{\True}{\code v_1}{\code v_2} \to \code v_1$} \DisplayProof
  \quad
  \AxiomC{$\IfThenElse{\False}{\code v_1}{\code v_2} \to \code v_2$} \DisplayProof
  \\ \vspace{1em}
  \AxiomC{$\code t_2 \to \code t'_2$}
  \UnaryInfC{$\IfThenElse{\code t_1}{\code t_2}{\code t_3} \to
              \IfThenElse{\code t_1}{\code t'_2}{\code t_3}$} \DisplayProof
  \\ \vspace{1em}
  \AxiomC{$\code t_2 \to \code t'_2$}
  \UnaryInfC{$\IfThenElse{\code t_1}{\code v}{\code t_2} \to
              \IfThenElse{\code t_1}{\code v}{\code t'_2}$} \DisplayProof
\end{center}

\end{document}
